\chapter{Evaluation}
    %
    \section{Calibration of the Force Sensor}
        Because of a measurement misfortune for the calibration, the common way for determining the calibration factor is inoperable. Therefore another method is improvised. Originally, the calibration factor $ c $ was supposed to be determined by the linear fit of the ADC result-force-curve, while the weight of the different masses would have been converted into forces:
        \begin{align}
            c=\frac{\Delta n}{\Delta F} &&\left[\frac{1}{\SI{}{mN}}\right]
        \end{align}
        Since this measurement failed, the literature value of the surface tension of distilled water has to be used in order to be able to carry out a backward calculation. For that, \cref{eq:surface tension} is transformed into the force $ F_0 $:
        \begin{align*}
            F_0=\sigma \cdot 4\pi \cdot R_{ring}
        \end{align*}
        With
        \begin{align}
            F_0=F_{max}=\frac{n_{max}}{c}
            \label{eq:force}
        \end{align}
        it follows for distilled water:
        \begin{gather}
            \frac{n_{max}}{c} = \sigma_{H_2O} \cdot 4\pi \cdot R_{ring} \nonumber \\
            \Leftrightarrow \nonumber \\
            c = \frac{n_{max}}{\sigma_{H_2O} \cdot 4\pi \cdot R_{ring}}
            \label{eq:calibration factor}
        \end{gather}
        with \(\sigma_{H_2O}=\SI{72.8}{\frac{mN}{m}}\) at \SI{20}{\celsius} \cite{Eichler.2016} and \(R_{ring} = \SI{0.03}{m} \pm \SI{0.00005}{m}\)
        (approximated with the inner radius $ r_i=\SI{0.02923}{m} $ and outer radius $ r_o=\SI{0.03}{m} $)\par\medskip
        %
        The ADC result \(n_{max}\) can be read from \textsc{Du Noüy}s ring method measurements with distilled water (\cref{fig:du_nouy_method_measurement_with_distilled_water_no_1_for_calibration}).
        There are 10 values for \(n_{max}\):
        \begin{align*}
            n_{max,i}=[80321, 77600, 77085, 80827, 76744, 82423, 79322, 77319, 79543, 77873]
        \end{align*}
        The mean value and the standard deviation are
        \begin{align*}
            \bar{n}_{max}&=78906 \\
            \Delta n_{max}&=1883
        \end{align*}
        \Cref{eq:calibration factor} gives the calibration factor with
        \begin{align*}
            \boxed{c=\SI{(2875 \pm 73)}{\frac{1}{mN}}}
        \end{align*}
        The deviation of the calibration factor is calculated as follows:
        \begin{align}
            \Delta c    &=\left| \frac{\partial c}{\partial n_{max}} \right| \cdot \Delta n_{max} + \left| \frac{\partial c}{\partial R_{Ring}} \right| \cdot \Delta R_{Ring} \nonumber \\
                        &=\frac{\Delta n_{max}}{\sigma_{H_2O} \cdot 4\pi \cdot R_{Ring}} + \frac{n_{max} \cdot \Delta R_{Ring}}{\sigma_{H_2O} \cdot 4\pi \cdot R_{Ring}^2} \nonumber \\
                        &=\frac{1883}{\SI{72.8}{\frac{mN}{m}} \cdot 4\pi \cdot \SI{0.03}{m}} + \frac{78906 \cdot 5 \cdot \SI{10^{-5}}{m}}{\SI{72.8}{\frac{mN}{m}} \cdot 4\pi \cdot (\SI{0.03}{m})^2} \nonumber \\
                        &=\SI{68.61}{\frac{1}{mN}}+\SI{4.79}{\frac{1}{mN}} \nonumber \\
                        &=\SI{73.4}{\frac{1}{mN}} \approx \SI{73}{\frac{1}{mN}}
        \end{align}
        \begin{figure}[h]
            \centering
            \includegraphics[width=.9\textwidth]{scidavis/Du_Nouy_Method_Measurement_with_distilled_water_No_1_for_cal.jpg}
            \caption[]{}
            \label{fig:du_nouy_method_measurement_with_distilled_water_no_1_for_calibration}
        \end{figure}
        %
    \section{Resolution and Statistics}
        For comparison of the raw and filtered data of the ADC with regard to statistical variations, both are
        plotted as a function of the time in a stray diagram in \cref{fig:adc_result}.\par
        As it can be seen, the raw data has a greater scattering. This is also confirmed by the standard deviations, which are as follows:
        \begin{figure}[h]
            \centering\includegraphics[width=.9\textwidth]{scidavis/ADC_result.jpg}
            \caption[]{}
            \label{fig:adc_result}
        \end{figure}
        \begin{align*}
            \bar{n}_{raw}       &=-21 \qquad \sigma_{raw}=58\\
            \bar{n}_{filtered}  &=-23 \qquad \sigma_{filtered}=39
        \end{align*}
        Based on that, the deviation of the force to be read can be determined as
        \begin{align}
            \Delta F    &=\left| \frac{\partial F}{\partial \bar{n}_{filtered}} \right| \cdot \sigma_{filtered} + \left| \frac{\partial F}{\partial c} \right| \cdot \Delta c \nonumber \\
                        &=\frac{1}{c} \cdot \sigma_{filtered} + \frac{\left|\bar{n}_{filtered}\right|}{c^2} \cdot \Delta c \nonumber \\
                        &=\frac{39}{\SI{2875}{\frac{1}{mN}}} + \frac{23 \cdot \SI{73}{\frac{1}{mN}}}{(\SI{2875}{\frac{1}{mN}})^2} \cdot \SI{73}{\frac{1}{mN}} \nonumber \\
                        &=\SI{0.0135}{mN}+\SI{0.0002}{mN} \nonumber \\
                        &=\SI{13.7}{\micro N}
        \end{align}
        The effective resolution of the ADC is given by means of \cref{eq:ENOB}:
        \begin{align*}
            ENOB&=-23-\log_2(39)=-28
        \end{align*}
        Further, the histograms (\cref{fig:histogram_raw_data} and \cref{fig:histogram_filtered_data}) show that the raw
        data is more widely spread than the filtered one. The raw data histogram has a binning value of around 40 and the
        filtered data one of around 20.
        \begin{figure}[H]
            \centering
            \includegraphics[width=.9\textwidth]{scidavis/histogram_filtered_data.jpg}
            \caption[]{}
            \label{fig:histogram_filtered_data}
        \end{figure}
        \begin{figure}[H]
            \centering
            \includegraphics[width=.9\textwidth]{scidavis/histogram_raw_data.jpg}
            \caption[]{}
            \label{fig:histogram_raw_data}
        \end{figure}
        %
    \section{\textsc{Du Noüy} Ring Method Measurement}
        The data is being sent to the PC during the measurement procedure. The diagram representing force over time, as it
        is displayed in \cref{fig:Du_Nouy_Method_Measurement_with_distilled_water_No_1,fig:Du_Nouy_Method_Measurement_with_detergent_No_1,fig:Du_Nouy_Method_Measurement_with_more_detergent_No_1,fig:Du_Nouy_Method_Measurement_with_isopropanol_No_1}, is obtained converting the ADC data into the force by way of \cref{eq:force}. 
        %
        \subsection*{Distilled Water}
            The maximum forces for distilled water are as follows:
            \begin{align*}
                F_{max,i}^{H_2O}=[27.9, 27.0, 26.8, 28.1, 26.7, 28.7, 27.6, 26.9, 27.7, 27.1] && [\SI{}{mN}]
            \end{align*}
            The forces are around the mean value
            \begin{align*}
                \bar{F}_{max}^{H_2O}=\SI{(27.5 \pm 0.7)}{mN}
            \end{align*}
            so the calculated value in \cref{eq:calculated force} is approximated well. By using \cref{eq:surface tension}
            for the surface tension of distilled water, it results:
            \begin{align*}
                \boxed{\sigma_{H_2O}^{det}=\SI{(72.95 \pm 0.15)}{\frac{mN}{m}}}
            \end{align*}
            with a deviation
            \begin{align}
                \Delta \sigma_{H_2O}^{det}  &= \left| \frac{\partial \sigma_{H_2O,determined}}{\partial \bar{F}_{max,H_2O}} \right| \cdot \Delta F + \left| \frac{\partial \sigma_{H_2O,determined}}{\partial R} \right| \cdot \Delta R \nonumber \\
                                            &= \frac{\Delta F}{4\pi \cdot R_{ring}} + \frac{\bar{F}_{max}^{H2O}\cdot \Delta R_{ring}}{4\pi \cdot R_{ring}^2} \nonumber \\
                                            &= \frac{\SI{0.01}{mN}}{4\pi \cdot \SI{0.03}{m}} + \frac{\SI{27.5}{mN}}{4\pi \cdot (\SI{0.03}{m})^2} \cdot \SI{0.00005}{m} \nonumber \\
                                            &= \SI{0.0265}{\frac{mN}{m}} + \SI{0.122}{\frac{mN}{m}} \nonumber \\
                                            &= \SI{148.5}{\frac{\micro N}{m}} \approx \SI{150}{\frac{\micro N}{m}}
            \end{align}
            The determined value of $ \SI{72.95}{\frac{mN}{m}} $ fits well with the literature value of $ \SI{72.8}{\frac{mN}{m}} $. The area of error covers the literature value.
            %
        \subsection*{Detergent}
            The maximum forces for detergent are
            \begin{align*}
                F_{max,i}^{detergent} = [26.8, 25.8, 26.4, 23.9, 21.0, 21.6, 23.0, 22.6, 27.9, 23.8] &&[\SI{}{mN}]
            \end{align*}
            and their mean value is
            \begin{align*}
                \bar{F}_{max}^{detergent}=\SI{(24.3 \pm 2.3)}{mN}
            \end{align*}
            For the detergents surface tension it results:
            \begin{align}
                \boxed{\sigma_{detergent}=\SI{(64.46 \pm 0.13)}{\frac{mN}{m}}}
            \end{align}
            The deviation is calculated equivalently to the one of distilled water.\\
            In order to be able to compare this value with the literature value, the concentration used has to be estimated.
            Approximately, the used mass of sodium dodecyl sulfate (detergent) is about \(\SI{0.01}{g}\). Its molar mass
            is \(\SI{288.4}{\frac{g}{mol}}\) \cite{sodium.dodecyl.sulfate.cas.entry.2021}. For the amount of substance it follows:
            \begin{align*}
                \text{amount of substance} = \frac{\text{mass}}{\text{molar mass}} = \frac{\SI{0.01}{g}}{\SI{288.4}{\frac{g}{mol}}} = \SI{0.035}{mmol}
            \end{align*}
            Thus the concentration for $ \SI{100}{mL} $ volume of water is
            \begin{align*}
                \text{concentration} = \frac{\text{amount of substance}}{\text{volume}} = \frac{\SI{0.035}{mmol}}{\SI{100}{mL}} = \SI{0.35}{\frac{mmol}{L}}
            \end{align*}
            In \cite{Quelle1} at $ \SI{0.35}{\frac{mmol}{L}} $ a surface tension of about $ \SI{63}{\frac{mN}{m}} $ can be read. So there is a good match between determined and literature value.

        \subsection*{More Detergent}
            Adding more detergent leads to these maximum forces:
            \begin{align*}
                F_{max,more detergent,i}=[25.8, 24.9, 20.5, 20.8, 18.5, 18.8, 20.9, 26.7, 28.0, 27.6] \quad \text{(all values in \SI{}{mN})}
            \end{align*}
            with the mean value
            \begin{align*}
                \bar{F}_{max,more detergent}=\SI{23.3}{mN}
            \end{align*}
            This means the following surface tension:
            \begin{align*}
                \boxed{\sigma_{more detergent}=\SI{(61.81 \pm 0.13)}{\frac{mN}{m}}}
            \end{align*}
            The surface tension of the liquid with added detergent is a little bit smaller than before. In \cite{synth.of.ACD.as.surfactant.Kumar.2015} it can be seen that the surface tension decreases with increasing concentration.

        \subsection*{Isopropanol}
            With isopropanol the following maximum forces are obtained:
            \begin{align*}
                F_{max,isopropanol,i}=[8.6, 8.0, 5.9, 5.9, 6.3, 7.7, 5.6, 7.9, 7.4, 7.9] \quad \text{(all values in \SI{}{mN})}
            \end{align*}
            They give the mean value
            \begin{align*}
                \bar{F}_{max,isopropanol}=\SI{6.3}{mN}
            \end{align*}
            The determined surface tension of isopropanol is therefore
            \begin{align*}
                \boxed{\sigma_{isopropanol}=\SI{(16.71 \pm 0.05)}{\frac{\milli\newton}{\metre}}}
            \end{align*}
            The deviation is also calculated as above.\\
            Both the determined and the literature value of isopropanols surface tension are in the same magnitude. The literature value is $ \SI{21.4}{\frac{mN}{m}} $ (\cite{Eichler.2016}) and thus the determined one is slightly lower. Here, the literature value is not covered by the error area as well.
