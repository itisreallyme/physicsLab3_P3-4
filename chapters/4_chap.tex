\chapter{Evaluation}
%
    \section{Calibration of the Force Sensor}
        Because of a measurement misfortune for the calibration, the common way for determining the calibration factor is
        inoperable. Therefore, another method is improvised. Originally, the calibration factor \(c\) was supposed to be
        determined by the linear fit of the ADC result-force-curve, while the weight of the different masses would have
        been converted into forces:
        \begin{align}
            c=\frac{\Delta n}{\Delta F}, \qquad [c]=\frac{1}{\SI{}{mN}}
        \end{align}
        Since this measurement failed, the literature value of the surface tension of distilled water has to be used in
        order to be able to carry out a backward calculation. For that, \cref{eq:surface tension} is transformed into the
        force \(F_0\):
        \begin{align*}
            F_0=\sigma \cdot 4\pi \cdot R_{Ring}
        \end{align*}
        With
        \begin{align}
            F_0=F_{max}=\frac{n_{max}}{c}
            \label{eq:force}
        \end{align}
        it follows for distilled water:
        \begin{gather}
            \frac{n_{max}}{c}           =\sigma_{H_2O} \cdot 4\pi \cdot R_{Ring} \nonumber \\
            \Leftrightarrow \nonumber\\
            c                           =\frac{n_{max}}{\sigma_{H_2O} \cdot 4\pi \cdot R_{Ring}}
            \label{eq:calibration factor}
        \end{gather}
        with \(\sigma_{H_2O}=\SI{72.8}{\frac{mN}{m}}\) at \SI{20}{\celsius} \cite{Eichler.2016}.
        \par\medskip
        The ADC result \(n_{max}\) can be read from the \textsc{Du Noüy}-Ring method measurements with distilled water. There are
        10 values for \(n_{max}\):
        \begin{align*}
            n_{max,i}=[80321, 77600, 77085, 80827, 76744, 82423, 79322, 77319, 79543, 77873]
        \end{align*}
        The mean value and the deviation (as standard deviation) are
        \begin{align}
            \bar{n}_{max}   &=78906 \nonumber \\
            \Delta n_{max}  &=1883
        \end{align}
        The deviation of the calibration factor is calculated as follows:
        \begin{align}
            \Delta c&=\left| \frac{\partial c}{\partial n_{max}} \right| \cdot \Delta n_{max} + \left| \frac{\partial c}{\partial R_{Ring}} \right| \cdot \Delta R_{Ring} \nonumber \\
            &=\frac{\Delta n_{max}}{\sigma_{H_2O} \cdot 4\pi \cdot R_{Ring}} + \frac{n_{max} \cdot \Delta R_{Ring}}{\sigma_{H_2O} \cdot 4\pi \cdot R_{Ring}^2} \nonumber \\
            &=\frac{1883}{\SI{72.8}{\frac{mN}{m}} \cdot 4\pi \cdot \SI{0.03}{m}} + \frac{78906 \cdot \SI{50 \cdot 10^{-6}}{m}}{\SI{72.8}{\frac{mN}{m}} \cdot 4\pi \cdot (\SI{0.03}{m})^2} \nonumber \\
            %&=\SI{68.61}{\frac{1}{mN}}+\SI{4.79}{\frac{1}{mN}} \nonumber \\
            &=\SI{73.4}{\frac{1}{mN}}% \approx \SI{73}{\frac{1}{mN}}
            \label{eq:calibration factor deviation}
        \end{align}
        \Cref{eq:calibration factor} and \cref{eq:calibration factor deviation} give a calibration factor of
        \begin{align}
            \fbox{c=\SI{(2875 \pm 73)}{\frac{1}{mN}}}
        \end{align}
        %
    \section{Resolution and Statistics}
        For comparison of the raw and filtered data of the ADC with regard to statistical variations, both are
        plotted as a function of the time in a stray diagram.

        As it can be seen, the raw data has a greater scattering. This is also confirmed by the standard deviations, which
        are as follows:
        \begin{align*}
            \bar{n}_{raw}       &=-21, \qquad \sigma_{raw}=58\\
            \bar{n}_{filtered}  &=-23, \qquad \sigma_{filtered}=39
        \end{align*}
        Based on that, the deviation of the force to be read can be determined as
        \begin{align}
            \Delta F    &=\left| \frac{\partial F}{\partial \bar{n}_{filtered}} \right| \cdot \sigma_{filtered} + \left| \frac{\partial F}{\partial c} \right| \cdot \Delta c \nonumber \\
                        &=\frac{1}{c} \cdot \sigma_{filtered} + \frac{\left|\bar{n}_{filtered}\right|}{c^2} \cdot \Delta c \nonumber \\
                        &=\frac{39}{\SI{2875}{\frac{1}{mN}}} + \frac{23 \cdot \SI{73}{\frac{1}{mN}}}{(\SI{2875}{\frac{1}{mN}})^2} \nonumber \\
                        %&=\SI{13.5}{\micro N} + \SI{0.2}{\micro N} \nonumber \\
                        &=\SI{0.0137}{mN}% \approx \SI{0.01}{mN}
        \end{align}
        The effective resolution of the ADC is given by means of \cref{eq:ENOB}:
        \begin{align*}
            ENOB    &=-23-\log_2(39)=-28
        \end{align*}
        Further, the histograms show that the raw data is more widely spread than the filtered one. The raw data histogram
        has a binning value of around 40 and the filtered data one of around 20.
        %
    \section{Du Noüy Ring Method Measurement}
        The data is being sent to the PC during the measurement procedure. When the ADC results are converted into the
        force by way of \cref{eq:force}, the force-time-diagram is obtained.

        The maximum forces for distilled water are as follows:
        \begin{align*}
            F_{max,H_2O,i}=[27.9, 27.0, 26.8, 28.1, 26.7, 28.7, 27.6, 26.9, 27.7, 27.1] \quad \text{(all values in \SI{}{mN})}
        \end{align*}
        The forces are around the mean value
        \begin{align}
            \bar{F}_{max,H_2O}=\SI{27.5}{mN}
        \end{align}
        so the calculated value in \cref{eq:calculated force} is approximated well.
